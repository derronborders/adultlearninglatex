%%%%%%%%%%%%%%%
% This CV example/template is based on my own
% CV which I (lamely attempted) to clean up, so that
% it's less of an eyesore and easier for others to use.
%
% LianTze Lim (liantze@gmail.com)
% 23 Oct, 2022
%
\documentclass[letter,skipsamekey,11pt,english]{curve}


% Uncomment to enable Chinese; needs XeLaTeX
% \usepackage{ctex}


% Default biblatex style used for the publication list is APA6. If you wish to use a different style or pass other options to biblatex you can change them here. 


% Most commands and style definitions are in settings.sty.
\usepackage{settings}
\usepackage{tabularx}
\usepackage[none]{hyphenat}


\usepackage{fancyhdr}
\pagestyle{fancy}
\setlength{\headheight}{14pt}
\renewcommand{\headrulewidth}{0pt}
\renewcommand{\footrulewidth}{0pt}

\fancyfoot[COE]{ \fontsize{8}{11}\color[HTML]{999999} \today \hfill  Borders Curriculum Vit\ae \hfill \thepage}



\makeatletter
\RenewDocumentCommand{\@sentry}{O{} O{\faBookmark}}{%
  \gdef\@nextentry{\\\par}\@key{#1}\egroup
  \@@key & \prefixmarker{#2}\@prefix &}
\makeatother

%% Only needed if you want a Publication List
\addbibresource{own-bib.bib}



%% Specify your last name(s) and first name(s) (as given in the .bib) to automatically bold your own name in the publications list. 
%% One caveat: You need to write \bibnamedelima where there's a space in your name for this to work properly; or write \bibnamedelimi if you use initials in the .bib
% \mynames{Lim/Lian\bibnamedelima Tze}

%% You can specify multiple names like this, especially if you have changed your name or if you need to highlight multiple authors. See items 6–9 in the example "Journal Articles" output.
\mynames{Borders\bibnamedelima Derron,
  Borders\bibnamedelima Derron,
  Borders/Derron,
  Borders/B.\bibnamedelimi D.}
%% MAKE SURE THERE IS NO SPACE AFTER THE FINAL NAME IN YOUR \mynames LIST


% Change the fonts if you want
\ifxetexorluatex % If you're using XeLaTeX or LuaLaTeX
  \usepackage{fontspec} 
  %% You can use \setmainfont etc; I'm just using these font packages here because they provide OpenType fonts for use by XeLaTeX/LuaLaTeX anyway
  \usepackage[p,osf,swashQ]{cochineal}
  \usepackage[medium,bold]{cabin}
  \usepackage[varqu,varl,scale=0.9]{zi4}
\else % If you're using pdfLaTeX or latex
  \usepackage[T1]{fontenc}
  \usepackage[p,osf,swashQ]{cochineal}
  \usepackage{cabin}
  \usepackage[varqu,varl,scale=0.9]{zi4}
\fi

% Change the page margins if you want
% \geometry{left=1cm,right=1cm,top=1.5cm,bottom=1.5cm}

% Change the colours if you want
% \definecolor{SwishLineColour}{HTML}{00FFFF}
% \definecolor{MarkerColour}{HTML}{0000CC}

% Change the item prefix marker if you want
% \prefixmarker{$\diamond$}

%% Photo is only shown if "fullonly" is included
\includecomment{fullonly}
% \excludecomment{fullonly}


%%%%%%%%%%%%%%%%%%%%%%%%%%%%%%%%%%%%%%


\definecolor{SwishLineColour}{HTML}{512888}
\definecolor{MarkerColour}{HTML}{512888}






\title{Curriculum Vit\ae}

\begin{document}
\begin{center}
  \textcolor{MarkerColour}{\Huge\sffamily\scshape\bfseries Derron Borders}\\
~\\
\textcolor{MarkerColour}{\Large\sffamily\scshape Kansas State University, Manhattan, KS}\\
\vspace{1em}

\begin{tabularx}{1\textwidth} { 
   >{\raggedright\arraybackslash}X 
    >{\raggedright\arraybackslash}X  }

 \makefield{\faHome} {111 Fake St., Manahattan, Ks 66502}  & \hspace{3cm}
 \makefield{\faGlobe} {\url{www.websitename.com}} \\

 \makefield{\faEnvelope[regular]}   {\href{mailto:email}{\texttt{email@email.edu}}}    & \hspace{3cm} \makefield{\faLinkedin}
  {\href{http://www.linkedin.com/in/linkedinid/}{\texttt{linkedinid}}}  \\
 
 \makefield{\faMobile*} {phone number} &\hspace{3cm} \makefield{\faTwitter}{\href{https://x.com/xname}{\texttt{@xusername}}}\\

\end{tabularx}
\end{center}


\makerubric{cv/research}

\makerubric{cv/education}


\makerubric{cv/administrative}
\makerubric{cv/industry}
\makerubric{cv/assistantships}
\makerubric{cv/teaching}
% If you're not a researcher nor an academic, you probably don't have any publications; delete this line.
%% Sometimes when a section can't be nicely modelled with the \entry[]... mechanism; hack our own and use \input NOT \makerubric
%% Sometimes when a section can't be nicely modelled with the \entry[]... mechanism; hack our own
\makerubrichead{Publications}

%% Assuming you've already given \addbibresource{own-bib.bib} in the main doc. Right? Right???
\nocite{*}

%% If you just want everything in one list
%\printbibliography

\printbibliography[heading={subbibliography},title={Theses},type=masterthesis]

\printbibliography[heading={subbibliography},title={Unpublished Manuscripts},keyword={unpublished}]

\printbibliography[heading={subbibliography}, title={Scholarly Paper Presentations (Refereed)},keyword={scholarlypresentation}]

\printbibliography[heading={subbibliography}, title={Conference Presentations (Refereed)},keyword={conference}]

\printbibliography[heading={subbibliography}, title={Invited Lectures},keyword={lecture}]

\printbibliography[heading={subbibliography}, title={Invited Talks},keyword={invited}]

\printbibliography[heading={subbibliography}, title={Selected Workshops (Facilitator)},keyword={workshop}]



%\printbibliography[heading={subbibliography},title={Journal Articles},type=article]


\makerubric{cv/universityservice}
\makerubric{cv/nationalservice}
\makerubric{cv/awards}
\makerubric{cv/skills}
%\makerubric{cv/misc}

%\makerubric{cv/referee}
% \input{referee-full}

\end{document}